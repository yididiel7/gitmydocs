% Options for packages loaded elsewhere
\PassOptionsToPackage{unicode}{hyperref}
\PassOptionsToPackage{hyphens}{url}
\documentclass[
]{article}
\usepackage{xcolor}
\usepackage{amsmath,amssymb}
\setcounter{secnumdepth}{-\maxdimen} % remove section numbering
\usepackage{iftex}
\ifPDFTeX
  \usepackage[T1]{fontenc}
  \usepackage[utf8]{inputenc}
  \usepackage{textcomp} % provide euro and other symbols
\else % if luatex or xetex
  \usepackage{unicode-math} % this also loads fontspec
  \defaultfontfeatures{Scale=MatchLowercase}
  \defaultfontfeatures[\rmfamily]{Ligatures=TeX,Scale=1}
\fi
\usepackage{lmodern}
\ifPDFTeX\else
  % xetex/luatex font selection
\fi
% Use upquote if available, for straight quotes in verbatim environments
\IfFileExists{upquote.sty}{\usepackage{upquote}}{}
\IfFileExists{microtype.sty}{% use microtype if available
  \usepackage[]{microtype}
  \UseMicrotypeSet[protrusion]{basicmath} % disable protrusion for tt fonts
}{}
\makeatletter
\@ifundefined{KOMAClassName}{% if non-KOMA class
  \IfFileExists{parskip.sty}{%
    \usepackage{parskip}
  }{% else
    \setlength{\parindent}{0pt}
    \setlength{\parskip}{6pt plus 2pt minus 1pt}}
}{% if KOMA class
  \KOMAoptions{parskip=half}}
\makeatother
\setlength{\emergencystretch}{3em} % prevent overfull lines
\providecommand{\tightlist}{%
  \setlength{\itemsep}{0pt}\setlength{\parskip}{0pt}}
\usepackage{bookmark}
\IfFileExists{xurl.sty}{\usepackage{xurl}}{} % add URL line breaks if available
\urlstyle{same}
\hypersetup{
  hidelinks,
  pdfcreator={LaTeX via pandoc}}

\author{}
\date{}

\begin{document}

\textbf{Title:} Introduction to Command-Line Interfaces (CLIs) on Arch
Linux \textbf{Author:} Sr.~Yididiel Hills Arch Linux AMD, Developer
\textbf{Date:} November, 15-11-2025 (20:57:02 IST) \textbf{lfnc:}
Beginner\_Tutorial\_Arch0\_Intro\_CLI.md \textbf{Copyright:} ©
2025--2026 Sr.~Tyrone Hills \textbf{License:} MIT -- see LICENSE file
for details

Welcome to this comprehensive DIY tutorial on installing and configuring
Command-Line Interfaces (CLIs) on Arch Linux.

In this guide, we will walk you through the process of setting up a CLI
environment on your Arch Linux system.

\subsection{Required Dependencies}\label{required-dependencies}

\begin{itemize}
\tightlist
\item
  \texttt{curl} package
\item
  \texttt{ssh-agent} package
\item
  \texttt{sshd} service package
\item
  \texttt{pwgen} command-line utility
\end{itemize}

\subsection{Estimated Time}\label{estimated-time}

This tutorial should take approximately 1-2 hours to complete, depending
on your familiarity with CLI basics and the complexity of your Arch
Linux installation.

\subsection{Skill Level}\label{skill-level}

\begin{itemize}
\tightlist
\item
  Beginner: This tutorial assumes a basic understanding of Linux
  commands and file systems. If you are new to Linux or CLIs, we
  recommend starting with a more comprehensive guide before attempting
  this tutorial.
\item
  Intermediate: If you have some experience with Linux and CLIs, you
  will likely find this tutorial easier to follow.
\item
  Advanced: If you are already familiar with CLI basics and advanced
  system administration tasks, you may be able to complete this tutorial
  quickly.
\end{itemize}

\subsection{Step-by-Step Instructions}\label{step-by-step-instructions}

\subsubsection{Step 1: Install Required
Dependencies}\label{step-1-install-required-dependencies}

\begin{itemize}
\tightlist
\item
  Open a terminal on your Arch Linux system.
\item
  Run the following command to install \texttt{curl},
  \texttt{ssh-agent}, \texttt{sshd}, and \texttt{pwgen} packages:
\end{itemize}

\begin{verbatim}
sudo pacman -S curl ssh-agent sshd pwgen
\end{verbatim}

\subsubsection{Step 2: Generate SSH
Keys}\label{step-2-generate-ssh-keys}

\begin{itemize}
\tightlist
\item
  Create a new SSH key pair using the following command:
\end{itemize}

\begin{verbatim}
ssh-keygen -t rsa -b 4096
\end{verbatim}

\subsubsection{Step 3: Add SSH Key to
System}\label{step-3-add-ssh-key-to-system}

\begin{itemize}
\tightlist
\item
  Add the generated SSH key to your system by running the following
  command:
\end{itemize}

\begin{verbatim}
ssh-agent -s
ssh-add ~/.ssh/id_rsa
\end{verbatim}

\subsubsection{Step 4: Create SSH User}\label{step-4-create-ssh-user}

\begin{itemize}
\tightlist
\item
  Create a new user with SSH capabilities using the following command:
\end{itemize}

\begin{verbatim}
useradd -m sshuser
\end{verbatim}

\subsubsection{Step 5: Generate Password for SSH
User}\label{step-5-generate-password-for-ssh-user}

\begin{itemize}
\tightlist
\item
  Generate a password for the new SSH user using the following command:
\end{itemize}

\begin{verbatim}
pwgen -s -1
\end{verbatim}

\subsubsection{Step 6: Configure SSH
Service}\label{step-6-configure-ssh-service}

\begin{itemize}
\tightlist
\item
  Create a new file called \texttt{sudoers} in the \texttt{/etc/ssh/}
  directory with the following contents:
\end{itemize}

\begin{verbatim}
# /etc/sudoers

Content of /etc/sudoers:

all ALL=(ALL) NOPASSWD:/home/sshuser/bin
all ALL=(ALL) NOPASSWD: /home/sshuser/sbin
\end{verbatim}

\subsubsection{Step 7: Set SSHD to Start at Boot
Time}\label{step-7-set-sshd-to-start-at-boot-time}

\begin{itemize}
\tightlist
\item
  Create a new file called \texttt{rc.d\_sshd} in the
  \texttt{/etc/init.d/} directory with the following contents:
\end{itemize}

\begin{verbatim}
# /etc/init.d/rc.d_sshd

Content of /etc/init.d/rc.d_sshd:

chown sshuser:root /etc/init.d/rc.d_sshd
chmod 644 /etc/init.d/rc.d_sshd
\end{verbatim}

\subsubsection{Step 8: Start SSHD at Boot
Time}\label{step-8-start-sshd-at-boot-time}

\begin{itemize}
\tightlist
\item
  Create a new file called \texttt{systemd/system/ssh.service} in the
  \texttt{/etc/systemd/system/} directory with the following contents:
\end{itemize}

\begin{verbatim}
# /etc/systemd/system/ssh.service

Content of /etc/systemd/system/ssh.service:

[Unit]

Description=SSH Service

After=network.target

[Service]

User=sshuser
Group=sshuser
ExecStart=/usr/bin/sshd -F
Restart=always

[Install]

WantedBy=multi-user.target
\end{verbatim}

\subsubsection{Step 9: Enable SSHD and Start at Boot
Time}\label{step-9-enable-sshd-and-start-at-boot-time}

\begin{itemize}
\tightlist
\item
  Reload the systemd daemon to pick up the new configuration:
\end{itemize}

\begin{verbatim}
sudo systemctl daemon-reload
\end{verbatim}

\begin{itemize}
\tightlist
\item
  Start the SSH service using the following command:
\end{itemize}

\begin{verbatim}
sudo systemctl start ssh
\end{verbatim}

\begin{itemize}
\tightlist
\item
  Enable the SSH service to start at boot time:
\end{itemize}

\begin{verbatim}
sudo systemctl enable ssh
\end{verbatim}

\subsubsection{Additional Tips and Troubleshooting
Advice}\label{additional-tips-and-troubleshooting-advice}

\begin{itemize}
\tightlist
\item
  Make sure to replace \texttt{sshuser} with your desired username.
\item
  You can customize the SSH key format by adding additional options to
  the \texttt{ssh-keygen} command. Refer to the official
  \href{https://www.openssh.com/docs.html}{SSH documentation} for more
  information.
\item
  If you encounter any issues during the installation process, refer to
  the Arch Linux forums or the official
  \href{https://wiki.archlinux.org/title/Installation_guide}{Arch Linux
  documentation} for troubleshooting advice.
\end{itemize}

By following these steps, you should now have a fully functional
Command-Line Interface on your Arch Linux system.

\end{document}
